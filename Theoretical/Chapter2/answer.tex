\documentclass[a4paper]{article}
\usepackage[affil-it]{authblk}
\usepackage{amsthm,amsmath,amssymb}
\usepackage{geometry}
\usepackage{hyperref}

\geometry{margin=1.5cm, vmargin={0pt,1cm}}
\setlength{\topmargin}{-1cm}
\setlength{\paperheight}{29.7cm}
\setlength{\textheight}{25.3cm}
\renewcommand{\qed}{\hfill \boxed{\mathbb{Q.E.D.}}}



\begin{document}
% =================================================
\title{Numerical Analysis Homework 2}

\author{Chen Wanqi 3220102895
  \thanks{Electronic address: \texttt{3220102895@zju.edu.cn}}}
\affil{Information and Computer Science 2201, Zhejiang University }


\date{\today}

\maketitle

% =============================================== 
\section*{Problem I.}

\textbf{Solution:}

First, recall that the linear interpolation \( p_1(f; x) \) at points \( x_0 \) and \( x_1 \) is given by Newton's formula:
\[
p_1(f; x) = f_0 + \frac{f_1 - f_0}{x_1 - x_0}(x - x_0),
\]
where \( f_0 = f(x_0) = 1 \) and \( f_1 = f(x_1) = \frac{1}{2} \). Substituting the values into the formula, we get:
\[
p_1(f; x) = 1 + \frac{\frac{1}{2} - 1}{2 - 1}(x - 1) = 1 - \frac{1}{2}(x - 1) = -\frac{1}{2}x + \frac{3}{2}.
\]

Next, the second derivative of \( f(x) \) is:
\[
f'(x) = -\frac{1}{x^2}, \quad f''(x) = \frac{2}{x^3}.
\]
Substituting these into the interpolation formula:
\[
\frac{1}{x} - \left( -\frac{1}{2}x + \frac{3}{2} \right) = \frac{f''(\xi(x))}{2} (x - 1)(x - 2),
\]
we simplify the left-hand side:
\[
\frac{1}{x} - \left( -\frac{1}{2}x + \frac{3}{2} \right) = \frac{2 - x}{2x}.
\]
Thus, the interpolation formula becomes:
\[
\frac{2 - x}{2x} = \frac{1}{\xi(x)^3} (x - 1)(x - 2).
\]

Finally, we obtain:
\[
\boxed{\xi(x) = \sqrt[3]{2x}.}
\]

Since \( \xi(x) \) is a monotonically increasing function on the interval \( x \in [1, 2] \), we can find the minimum and maximum values of \( \xi(x) \) on this interval:
\[
\boxed{\min \xi(x) = \xi(1) = \sqrt[3]{2}}, \quad \boxed{\max \xi(x) = \xi(2) = \sqrt[3]{4}.}
\]

Finally, the maximum value of \( f''(\xi(x)) \) occurs when \( \xi(x) \) is minimized. Since \( f''(x) = \frac{2}{x^3} \), we find:
\[
\max f''(\xi(x)) = f''(\min \xi(x)) = f''\left( \sqrt[3]{2} \right) = \frac{2}{\left( \sqrt[3]{2} \right)^3} = 1.
\]
Thus, \(\boxed{ \max f''(\xi(x)) = 1 }\).

%=======================================
\section*{Problem II.}

\textbf{Solution:}

Let 
\[
\ell_k(x) = \prod_{i=0, i \neq k}^{n} \frac{(x - x_i)^2}{(x_k - x_i)^2}.
\]
Clearly, \( \ell_k(x) \in \mathcal{P}_{2n}^+ \). And for every \( i \neq k \), we have \( \ell_k(x_i) = 0 \) and \( \ell_k(x_k) = 1 \).

Let
\[
P(x) = \sum_{k=0}^{n} f_k \ell_k(x),
\]
where \( P(x) \in \mathcal{P}_{2n}^+ \), and we can check that \( p(x_i) = f_i \), for \( i = 1, 2, \dots, n \).

\qed

%=======================================
\section*{Problem III.}

\textbf{Solution:}

For \( n = 0 \), clearly we have:
\[
f[t] = e^t.
\]
Assume the statement holds for \( n-1 \). By the induction hypothesis, we have:
\[
f[t, t+1, \dots, t+n] = \frac{f[t+1, t+2, \dots, t+n] - f[t, t+1, \dots, t+n-1]}{n}.
\]
Substituting the inductive assumption:
\[
f[t+1, t+2, \dots, t+n] = \frac{(e-1)^{n-1}}{(n-1)!} e^{t+1}, \quad f[t, t+1, \dots, t+n-1] = \frac{(e-1)^{n-1}}{(n-1)!} e^{t},
\]
we get:
\[
f[t, t+1, \dots, t+n] = \frac{\frac{(e-1)^{n-1}}{(n-1)!} e^{t+1} - \frac{(e-1)^{n-1}}{(n-1)!} e^t}{n} = \frac{(e-1)^n}{n!} e^t.
\]
Thus, the formula holds for all \( n \) by induction.

\qed

Now, substitute \( t = 0 \), we have:
\[
f[0, 1, \dots, n] = \frac{(e-1)^n}{n!}.
\]
From Corollary 2.22, we know that:
\[
f[0, 1, \dots, n] = \frac{f^{(n)}(\xi)}{n!},
\]
which implies:
\[
\xi = n \ln(e-1) \approx 0.541n.
\]
Since \( 0.541n > \frac{n}{2} \), \( \xi \) is located to the \boxed{right} of the midpoint \( \frac{n}{2} \).

%=======================================
\section*{Problem IV.}

\textbf{Solution:}
The table of divided differences is as follows:
\[
\begin{array}{c|cccc}
    0 & 5 &  &  &  \\
    1 & 3 & -2 &  &  \\
    3 & 5 & 1 & 1 &  \\
    4 & 12 & 7 & 2 & \frac{1}{4}
\end{array}
\]

By Newton's formula, we have:
\[
p_3(f; x) = 5 - 2x + (x)(x-1) + \frac{1}{4}(x)(x-1)(x-3) = \boxed{0.25x^3 - 2.25x + 5}.
\]

To find the approximate location of the minimum \( x_{\min} \), we differentiate \( p_3(f; x) \):
\[
p_3'(f; x) = 0.75x^2 - 2.25
\]
Setting \( p_3'(f; x) = 0 \) gives:
\[
\frac{3}{4} (x^2 - 3) = 0
\]

Thus, solving for \( x \), we find:
\[
x = \sqrt{3} \quad \text{(since \( x \) is limited within \( (1, 3) \))}.
\]
So the approximate minimum value is:
\[
\boxed{x_{\min} \approx \sqrt{3}}.
\]

%=======================================
\section*{Problem V.}

\textbf{Solution:}

Make the following table:

\begin{table}[htbp]
    \centering
    \begin{tabular}{c|cccccc}
        0 & 0 &  &  &  &  &  \\ 
        1 & 1 & 1 &  &  &  &  \\ 
        1 & 1 & 7 & 6 &  &  &  \\ 
        1 & 1 & 7 & 21 & 15 &  &  \\ 
        2 & 128 & 127 & 120 & 99 & 42 & \\ 
        2 & 128 & 448 & 321 & 201 & 102 & 30 \\ 
    \end{tabular}
\end{table}

Thus, we have \( \boxed{f[0, 1, 1, 1, 2, 2] = 30} \).

Next, we know that this divided difference can be expressed in terms of the 5th derivative of \( f \) evaluated at some \( \xi \in (0, 2) \). 

Since \( f(x) = x^7 \), we compute the 5th derivative:
\[
f^{(5)}(x) = 7 \times 6 \times 5 \times 4 \times 3 \times x^2 = 2520x^2.
\]

Now we set up the equation:
\[
2520\xi^2 = 30.
\]

Solving for \( \xi^2 \):
\[
\xi^2 = \frac{30}{2520} = \frac{1}{84}.
\]

Taking the square root:
\[
\xi = \sqrt{\frac{1}{84}} \approx 0.1091.
\]

Thus, we find that\(\boxed{\xi \approx 0.1091}\), which is located in the interval \( (0, 2) \).

%=======================================
\section*{Problem VI.}

\textbf{Solution:}

The divided differences table is constructed as follows:

\begin{table}[h]
    \centering
    \begin{tabular}{c|ccccc}
        0 & 0 &  &  &  &    \\
        1 & 2 & 1 &  &  &    \\
        1 & 2 & -1 & 2 &  &    \\
        3 & 0 & -1 & 0 & $\frac{2}{3}$ &    \\
        3 & 0 & 0 & $\frac{1}{2}$ & $\frac{1}{4}$ & $-\frac{5}{36}$ 
    \end{tabular}
\end{table}

This leads us to the Hermite polynomial expressed as:
\[
p(x)=1+x-2x(x-1)+\frac{2}{3}x(x-1)^2-\frac{5}{36}x(x-1)^2(x-3).
\]

Thus, we can approximate \( f(2) \) using \( p(2) \), yielding:
\[
\boxed{f(2) \approx p(2) = \frac{11}{18}}.
\]

According to Theorem 2.37, when substituting \( x=2 \), we obtain:
\[
\boxed{|f(x)-p(x)| = \left|\frac{f^{(5)}(\xi)}{5!}x(x-1)^2(x-3)^2\right| = \left|\frac{f^{(5)}(\xi)}{60}\right| \leq \frac{M}{60}}.
\]

%=======================================
\section*{Problem VII.}

\textbf{Proof:}

When \( n=1 \), it is clear that 
\[
\Delta^1 f(x) = \Delta f(x) = f(x+h) - f(x) = 1! h^1 \frac{f(x+h) - f(x)}{h} = 1! h^1 f[x, x+h].
\]

Assuming that the statement holds for \( n=k \) where \( k \in \mathbb{N}^* \), i.e., 
\[
\Delta^k f(x) = k! h^k f[x, x+h, \ldots, x+kh],
\]
we can derive the case for \( n=k+1 \):

\[
\begin{aligned}
\Delta^{k+1} f(x) & = \Delta^k f(x+h) - \Delta^k f(x) \\
& = k! h^k f[x+h, x+2h, \ldots, x+(k+1)h] - k! h^k f[x, x+h, \ldots, x+kh] \\
& = k! h^k \left( f[x+h, x+2h, \ldots, x+(k+1)h] - f[x, x+h, \ldots, x+kh] \right) \\
& = k! h^k (k+1) h f[x, x+h, \ldots, x+(k+1)h] \\
& = (k+1)! h^{k+1} f[x, x+h, \ldots, x+(k+1)h].
\end{aligned}
\]

Thus, when \( n=k+1 \), the result holds. By mathematical induction, we conclude that 
\[
\Delta^n f(x) = n! h^n f[x, x+h, \ldots, x+nh].
\]

Similarly, when \( n=1 \), it is clear that 
\[
\nabla^1 f(x) = \nabla f(x) = f(x) - f(x-h) = 1! h^1 \frac{f(x) - f(x-h)}{h} = 1! h^1 f[x, x-h].
\]

Assuming that the statement holds for \( n=k \) where \( k \in \mathbb{N}^* \), i.e., 
\[
\nabla^k f(x) = k! h^k f[x, x-h, \ldots, x-kh],
\]
we can similarly derive the case for \( n=k+1 \):

\[
\begin{aligned}
\nabla^{k+1} f(x) & = \nabla^k f(x) - \nabla^k f(x-h) \\
& = k! h^k f[x, x-h, \ldots, x-kh] - k! h^k f[x-h, x-2h, \ldots, x-(k+1)h] \\
& = k! h^k \left( f[x, x-h, \ldots, x-kh] - f[x-h, x-2h, \ldots, x-(k+1)h] \right) \\
& = k! h^k (k+1) h f[x, x-h, \ldots, x-(k+1)h] \\
& = (k+1)! h^{k+1} f[x, x-h, \ldots, x-(k+1)h].
\end{aligned}
\]

Thus, when \( n=k+1 \), the result holds. By mathematical induction, we conclude that 
\[
\nabla^n f(x) = n! h^n f[x, x-h, \ldots, x-nh].
\]

\qed


%=======================================
\section*{Problem VIII.}

\textbf{Proof:}

We start by using the definition of divided differences and their continuity:

\[
\begin{aligned}
    \frac{\partial}{\partial x_0} f[x_0, x_1, \dots, x_n] &= \lim_{h \to 0} \frac{f[x_0 + h, x_1, \dots, x_n] - f[x_0, x_1, \dots, x_n]}{h} \\
    &= \lim_{h \to 0} f[x_0, x_0 + h, x_1, \dots, x_n] \\
    &= f[x_0, x_0, x_1, \dots, x_n].
\end{aligned}
\]

This shows that taking the partial derivative of the divided difference with respect to \(x_0\) introduces an additional \(x_0\) into the sequence.

For the partial derivative with respect to any other variable, say \(x_i\) for \(i \neq 0\), the result follows a similar logic due to the symmetry of divided differences:

\[
\frac{\partial}{\partial x_i} f[x_0, x_1, \dots, x_n] = f[x_0, \dots, x_i, x_i, \dots, x_n].
\]

This means that differentiating with respect to any variable \(x_i\) duplicates that variable within the divided difference expression.

\qed

%=======================================
\section*{Problem IX.}

\textbf{Proof:}

We are tasked with solving the following min-max problem. For \( n \in \mathbb{N}^+ \) and a fixed \( a_0 \neq 0 \), we need to find:

\[
\min_{\{a_i \in \mathbb{R}\}} \max_{x \in [a, b]} \left| a_0 x^n + a_1 x^{n-1} + \dots + a_n \right|.
\]

We start by making the substitution:
\[
x = \dfrac{b-a}{2} x' + \dfrac{a+b}{2},
\]
so that \( x' \in [-1, 1] \). This transforms the original expression into:

\[
\min \max_{x' \in [-1, 1]} \left| a_0' {x'}^n + \dots + a_n' \right|.
\]

From this, using Corollary 2.47, we deduce the solution to be:

\[
\min \max_{x \in [a, b]} \left| a_0 x^n + \cdots + a_n \right| = \dfrac{1}{2^{n-1}} |a_0|.
\]

\qed

%=======================================
\section*{Problem X.}

\textbf{Proof:}

We are asked to prove that for the rescaled Chebyshev polynomial \( \hat{p}_n(x) \), the following inequality holds:

\[
\forall p \in P^a_n, \quad \|\hat{p}_n\|_\infty \leq \|p\|_\infty.
\]

We begin by considering the infinity norm of the Chebyshev polynomial:

\[
\| P_n(z) \|_\infty = \frac{| f_n(z) |_\infty}{| T_n(x) |_\infty} = \frac{1}{| T_n(x) |}.
\]

Suppose, for contradiction, that there exists a polynomial \( P \) such that \( \|P\|_\infty < \|P_n\|_\infty \). This would imply:

\[
\|P\|_\infty < \frac{1}{|T_n(x)|}.
\]

Define the difference between \( P \) and \( P_n \) as:

\[
r(n) = P(n) - P_n(n),
\]
which gives:
\[
r(x) = P(x) - P_n(x).
\]

Since \( r(x) = 0 \) at \( n+1 \) points, this leads to a contradiction. Hence, for all \( P \in P^a_n \), we conclude:

\[
\|\hat{p}_n\|_\infty \leq \|p\|_\infty.
\]

\qed

%=======================================
\section*{Problem XI.}

\textbf{Proof:}

The Bernstein base polynomial \( b_{n-1,k}(t) \) can be expressed as a linear combination of Bernstein polynomials of higher degrees. We begin by recalling the recursive definition of Bernstein polynomials:

\[
b_{n,k}(t) = \binom{n}{k} t^k (1 - t)^{n-k}, \quad t \in [0,1].
\]

Next, we expand \( b_{n-1,k}(t) \) as follows:

\[
b_{n-1,k}(t) = \binom{n-1}{k} t^k (1 - t)^{n-1-k}.
\]

By using the identity for binomial coefficients:
\[
\binom{n-1}{k} = \frac{n-k}{n} \binom{n}{k},
\]
we rewrite \( b_{n-1,k}(t) \) as:
\[
b_{n-1,k}(t) = \frac{n-k}{n} b_{n,k}(t).
\]

Similarly, for \( b_{n,k+1}(t) \), we use:
\[
\binom{n-1}{k+1} = \frac{k+1}{n} \binom{n}{k+1},
\]
which gives:
\[
b_{n-1,k}(t) = \frac{n-k}{n} b_{n,k}(t) + \frac{k+1}{n} b_{n,k+1}(t).
\]

Thus, we have expressed \( b_{n-1,k}(t) \) as a linear combination of \( b_{n,k}(t) \) and \( b_{n,k+1}(t) \), completing the proof:
\[
b_{n-1,k}(t) = \frac{n-k}{n} b_{n,k}(t) + \frac{k+1}{n} b_{n,k+1}(t).
\]

\qed

%=======================================
\section*{Problem XII.}

\textbf{Proof:}

We are tasked with proving that the integral of a Bernstein base polynomial \( b_{n,k}(t) \) over the interval \([0,1]\) depends only on its degree \( n \), i.e.,

\[
\forall k = 0,1,\dots,n, \quad \int_0^1 b_{n,k}(t) \, dt = \frac{1}{n+1}.
\]

Recall that the Bernstein base polynomial is defined as:

\[
b_{n,k}(t) = \binom{n}{k} t^k (1 - t)^{n-k}, \quad t \in [0,1], \, k = 0,1,\dots,n.
\]

We aim to compute the integral:

\[
I_{n,k} = \int_0^1 b_{n,k}(t) \, dt = \int_0^1 \binom{n}{k} t^k (1 - t)^{n-k} \, dt.
\]

Since the binomial coefficient \( \binom{n}{k} \) is constant with respect to \( t \), we factor it out of the integral:

\[
I_{n,k} = \binom{n}{k} \int_0^1 t^k (1 - t)^{n-k} \, dt.
\]

This integral is a standard Beta function, \( B(k+1, n-k+1) \), which is defined as:

\[
B(k+1, n-k+1) = \int_0^1 t^k (1 - t)^{n-k} \, dt.
\]

The Beta function has the following property in terms of Gamma functions:

\[
B(k+1, n-k+1) = \frac{\Gamma(k+1)\Gamma(n-k+1)}{\Gamma(n+2)}.
\]

Since \( \Gamma(k+1) = k! \) and \( \Gamma(n-k+1) = (n-k)! \), we get:

\[
B(k+1, n-k+1) = \frac{k! (n-k)!}{(n+1)!}.
\]

Thus, the integral becomes:

\[
I_{n,k} = \binom{n}{k} \cdot \frac{k! (n-k)!}{(n+1)!}.
\]

Simplifying using \( \binom{n}{k} = \frac{n!}{k!(n-k)!} \), we obtain:

\[
I_{n,k} = \frac{n!}{k!(n-k)!} \cdot \frac{k!(n-k)!}{(n+1)!} = \frac{1}{n+1}.
\]

Hence, the integral of a Bernstein base polynomial is independent of \( k \) and is given by:

\[
\int_0^1 b_{n,k}(t) \, dt = \frac{1}{n+1}, \quad k = 0,1,\dots,n.
\]

\qed

% =======================
\section*{References}
\begin{itemize}
   \item handoutsNumPDEs
   \item ChatGPT, AI Language Model, OpenAI Platform, 2024.
\end{itemize}
% ==============================================

\end{document}