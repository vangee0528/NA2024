\documentclass[a4paper]{article}
\usepackage[affil-it]{authblk}
\usepackage[backend=bibtex,style=numeric]{biblatex}

\usepackage{geometry}
\geometry{margin=1.5cm, vmargin={0pt,1cm}}
\setlength{\topmargin}{-1cm}
\setlength{\paperheight}{29.7cm}
\setlength{\textheight}{25.3cm}

\addbibresource{citation.bib}

\begin{document}
% =================================================
\title{Numerical Analysis homework  1}

\author{Chen Wanqi 3220102895
  \thanks{Electronic address: \texttt{3220102895@zju.edu.cn}}}
\affil{Information and Computer Science 2201, Zhejiang University }


\date{Due time: \today}

\maketitle





% ============================================
\section*{I.  Consider the bisection method starting with the initial interval [1.5, 3.5]. }

\subsection*{I-a. }
\textbf{Width of the interval at the nth step}

The bisection method halves the interval at each step.

For the interval $[1.5, 3.5]$, the initial width is $ 3.5 - 1.5 = 2 $. Therefore, the width at the $n$-th step is :  $W_n = \frac{2}{2^n} = \frac{1}{2^{n-1}}$
   

\subsection*{I-b.}
\textbf{Supremum of the distance between the root $r$ and the midpoint of the interval}

The distance between the root $r$ and the midpoint of the interval after $n$ steps, denoted as 
$D_n$, is always less than or equal to half the width of the interval. 

In the case of the interval $[1.5, 3.5]$, this becomes:  $D_n = \frac{2}{2^{n+1}} = \frac{1}{2^n}$
   
\section*{II. Briefly repeat the problem}

Give your answers here.





% ===============================================
\section*{ \center{\normalsize {Acknowledgement}} }
Give your acknowledgements here(if any).


\printbibliography

If you are not familiar with \texttt{bibtex}, 
it is acceptable to put a table here for your references.
\end{document}